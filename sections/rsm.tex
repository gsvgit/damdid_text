\section{RSM-based CFPQ algorithm}\label{section:RSM}

In this section, we provide background on Recursive State Machines (RSMs) and describe informally how they can specify context-free constraints for path querying in graphs. 
We also describe the main idea behind adopting GLL for an RSM-based CFPQ algorithm.

A \textit{recursive state machine} (RSM)~\cite{10.1145/1075382.1075387} provides a representation of context-free languages in a way resembling finite automata.
It allows us to use a graph-based specification of context-free languages and unify the processing of regular and context-free languages. 
Moreover, RSMs naturally encode grammars in Extended Backus-Naur Form (EBNF), which is often more compact and user-friendly than BNF.

In our work, we modify the GLL-based CFPQ algorithm to handle RSMs instead of grammars in BNF as a query specification.
This modification improves performance and enables native handling of both context-free and regular languages.
A detailed description of the algorithm, including formal definitions of descriptors, GSS, and correctness proofs, is beyond the scope of this paper and can be found in the extended version~\cite{abzalov2023gllbasedcontextfreepathquerying}.

Our solution is based on the generalized LL parsing algorithm~\cite{SCOTT2010177} which was shown to generalize well to graph processing~\cite{Grigorev:2017:CPQ:3166094.3166104}.
To support RSMs as a query specification, we introduce \emph{matched ranges}: each range is defined by two graph vertices (e.g., $u$ and $v$) and two RSM states (e.g., $p$ and $q$).
A matched range of the form $(p,u),(q,v)$ indicates that there exists at least one path from $u$ to $v$ such that its labels form a word accepted by RSM when starting in state $p$ and ending in state $q$.

As a result of parsing, GLL can construct a \textit{Shared Packed Parse Forest} (SPPF)~\cite{SCOTT20131828}, a special data structure which represents all possible derivations of the input in the compressed form.
A Shared Packed Parse Forest (SPPF) was proposed by Jan Rekers in~\cite{rekers1992parser} to represent parse forest without duplication of subtrees. Later, other versions of SPPF were introduced in different generalized parsing algorithms.
For example, GLL can provide the result in the form of \textit{binary subtree sets}~\cite{SCOTT201963}.
As shown in~\cite{Grigorev:2017:CPQ:3166094.3166104}, the SPPF can finitely represent the result of an all-paths query, enabling the recovery of any concrete path.
We employ SPPF with the following types of nodes. 

\begin{itemize}
    \item A \textit{Terminal node} to represent a matched edge label.
    \item A \textit{Nonterminal node} to represent the root of the subtree that corresponds to paths that can be derived from the respective nonterminal.  
    \item An \textit{Intermediate node} which corresponds to the intermediate point used to connect two matched ranges. It has exactly two children (both range nodes) and is denoted $I_{q,v}$, indicating that ranges are joined at point $(q,v)$.
    \item A \textit{Range node} corresponds to a matched range $(p,u),(q,v)$ and is denoted $R^{p,u}_{q,v}$. It encodes all possible derivations of that range and can have arbitrarily many children of any type except range. Range nodes are reusable within the SPPF.
    \item An \textit{Epsilon node} to represent the empty subtree in the case when nonterminal produces the empty string. 
\end{itemize}

\subsection{Example}\label{section:example_of_rsm}

We give a brief example.
Consider the graph as shown in Fig.~\ref{fig:input-graph} and we seek all paths from $v_0$ to $v_0$ that satisfy the constraint specified by the RSM in Fig.~\ref{fig:example-rsm}.
The RSM describes paths of the form $a^n b^n$, where $n > 0$.
Note that there are infinitely many such paths. 
Therefore, the resulting SPPF contain cycles.

\begin{figure}
    \centering
    
    \begin{tikzpicture}[->,>=stealth',shorten >=1pt,auto,node distance=2.8cm, semithick]

    \node[state] (A)                    {$v_0$};
    \node[state]         (B) [right of=A] {$v_1$};
    
    \path (A) edge  [loop left] node {a} (A)
                edge  [bend left] node {b} (B)
            (B) edge  [bend left] node {b} (A);
    \end{tikzpicture}

    \caption{Input graph.}
    \label{fig:input-graph}

\end{figure}

\begin{figure}
    \centering
    \begin{tikzpicture}[->,>=stealth',shorten >=1pt,auto,node distance=2.5cm, semithick]

    \node[initial,state]   (F)              {$q_4$};
    \node[state]           (G) [right of=F] {$q_5$};
    \node[accepting,state] (H) [right of=G] {$q_6$};
    \node[draw=black, fit= (F) (G) (H), inner sep=0.25cm] (J) {};
    \node[below right] at (J.north west) {S'};

    \path (F) edge              node {$q_0$} (G)
            (G) edge              node {$\$$} (H); 

    \node[initial,state] (A) [below = 2.7cm of F] {$q_0$};
    \node[state]         (B) [right of=A] {$q_1$};
    \node[state]         (D) [above right of=B] {$q_2$};
    \node[accepting,state]         (C) [right of=B] {$q_3$};
    \node [draw=black, fit= (A) (C) (D), inner sep=0.25cm] (E) {};
    \node[below right] at (E.north west) {S};

    \path (A) edge              node {a} (B)
            (B) edge              node {$q_0$} (D)
                edge              node {b} (C)
            (D) edge              node {b} (C);
    \end{tikzpicture}
    
    \caption{Extended RSM for grammar $S \to a \ b \mid a \ S \ b$. We automatically add the rule $S' \to S \ \$ $ to simplify query processing termination condition.}
    \label{fig:example-rsm}
\end{figure}

The resulting SPPF is shown in Fig.~\ref{fig:SPPF}.
Its root is the node $R^{q_4,v_0}_{q_5,v_0}$, which indicates that it represents paths from $v_0$ to $v_0$ that can be recognized by the RSM transitioning from state $q_4$ to $q_5$.
This transition corresponds to recognizing a word derived from the nonterminal $S$.
The node $R^{q_0,v_0}_{q_3,v_0}$ is the entry point for a cycle.
The node $R^{q_0,v_0}_{q_3,v_1}$ has two siblings, indicating two ways to construct the corresponding subpath.
The first corresponds to the shortest possible path: $$v_0 \xrightarrow{a} v_0 \xrightarrow{a} v_0 \xrightarrow{b} v_1 \xrightarrow{b} v_0.$$
The second contains the cycle mentioned above and therefore represents an infinite family of paths formed by repeatedly wrapping the shortest path with additional pairs of $a$s and $b$s.
For example, traversing the cycle once yields the path: $$v_0 \xrightarrow{a} v_0 \xrightarrow{a} v_0 \xrightarrow{a} v_0 \xrightarrow{a} v_0 \xrightarrow{b} v_1 \xrightarrow{b} v_0 \xrightarrow{b} v_1 \xrightarrow{b} v_0.$$

\begin{figure}[htbp] 
    \centering
    \resizebox{1\textwidth}{!}{
    \begin{tikzpicture}[->,>=stealth',shorten >=1pt,auto,node distance=1.6cm, semithick, every node/.style={font=\large}]

    \node[state]         (A)              {$R^{q_4,v_0}_{q_5,v_0}$};
    \node[state]         (B) [right of=A] {$S$};
    \node[state]         (C) [right of=B] {$R^{q_0,v_0}_{q_3,v_0}$};
    \node[state]         (D) [right of=C] {$I_{q_2,v_1}$};
    \node[state]         (E) [right of=D] {$R^{q_0,v_0}_{q_2,v_1}$};
    \node[state]         (F) [above of=E] {$R^{q_2,v_1}_{q_3,v_0}$}; 	
    \node[state]         (G) [right of=E] {$I_{q_1,v_0}$};
    \node[state]         (H) [right of=F] {$b$};
    \node[state]         (J) [right of=G] {$R^{q_0,v_0}_{q_1,v_0}$};
    \node[state]         (I) [above of=J] {$R^{q_1,v_0}_{q_2,v_1}$};
    \node[state]         (K) [below right of=J] {$a$};
    \node[state]         (L) [above right of=I] {$I_{q_1,v_0}$};
    \node[state]         (M) [above right of=L] {$S$};
    \node[state]         (N) [right of=M] {$R^{q_0,v_0}_{q_3,v_1}$};
    \node[state]         (O) [right of=N] {$I_{q_1,v_0}$};
    \node[state]         (P) [above of=O] {$I_{q_2,v_0}$};
    \node[state]         (R) [right of=P] {$R^{q_0,v_0}_{q_2,v_0}$};
    \node[state]         (S) [above of=R] {$R^{q_2,v_0}_{q_3,v_1}$};
    \node[state]         (T) [right of=O] {$R^{q_1,v_0}_{q_3,v_1}$};
    \node[state]         (U) [right of=T] {$b$};
    \node[state]         (V) [right of=S] {$b$};
    \node[state]         (W) [right of=R] {$I_{q_1,v_0}$};
    %\node[state]         (X) [right of=O] {$R^{q_0,v_0}_{q_1,v_0}$};
    \node[state]         (X) [above right of=W] {$R^{q_1,v_0}_{q_2,v_0}$};
    \node[state]         (Y) [above right of=X] {$S$};
    
    \path (A) edge          (B)
            (B) edge   (C)
            (C) edge   (D)
            (D) edge   (E)
            (D) edge   (F)
            (E) edge   (G)
            (F) edge   (H)
            (G) edge   (J)
            (G) edge   (I)
            (J) edge   (K)
            (I) edge   (L)
            (L) edge   (M)
            (M) edge   (N)
            (N) edge   (O)
            (N) edge   (P)
            (O) edge   (J)
            (O) edge   (T)
            (P) edge   (R)
            (R) edge   (W)
            (P) edge   (S)
            (T) edge   (U)
            (S) edge   (V)
            (W) edge[in=0, out=310]   (J)
            (W) edge   (X)
            (X) edge   (Y)
            (Y) edge[in=50, out=170]   (C)
            
            ;
    \end{tikzpicture}
    }
    \caption{The SPPF restored from the path index represented in~\ref{fig:sppf-matrix} for all paths from $v_0$ to $v_0$.}
    \label{fig:SPPF}
\end{figure}


