\section{Conclusion and Future Work}\label{section:conclusion}

In this work, we presented the GLL-based context-free path querying algorithm for the Neo4j graph database. The implementation is available on GitHub: \url{https://github.com/vadyushkins/cfpq-gll-neo4j-procedure}.

Our solution uses Neo4j for graph storage, but the query language should be extended to support context-free constraints to make it useful. Both the extension of Cypher and the integration of our algorithm with the query engine are non-trivial challenges left for future work.

While GLL-based CFPQ can potentially be used to solve the \textit{all-paths} problem, currently we have implemented the procedure only for the reachability. The choice of effective strategies to enumerate paths and implementation of them is a direction for future research.  

The most important direction for future work is to find a way to provide an incremental version of the GLL-based CFPQ algorithm to avoid full query reevaluation when the graph is only slightly changed. While our solution is based on the well-known parsing algorithm and there are solutions for incremental parsing, development of the efficient incremental version of the GLL-based CFPQ algorithm is a challenging problem left for future research. 

Another direction is to create a parallel version of the GLL-based CFPQ algorithm to improve its performance on huge graphs. Although it seems natural to handle descriptors in parallel, the algorithm operates over global structures, and the na{\"{\i}}ve implementation of this idea leads to a significant overhead in synchronization.