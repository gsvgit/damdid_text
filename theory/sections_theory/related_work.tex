\section{Related Work}\label{section:related_work}

The idea to use context-free grammars as constraints in the path finding problem in the graph databases was introduced and explored by Mihalis Yannakakis in~\cite{10.1145/298514.298576}. Later, Thomas Reps et al. developed the same idea into a general framework for static code analysis~\cite{10.1145/199448.199462}.
This framework, called Context-Free Language Reachability (CFL-r), is widely used and actively studied. The landscape analysis of the area was recently provided by Andreas Pavlogiannis in~\cite{10.1145/3583660.3583664}. In the context of the graph databases, the most recent systematic comparison of different CFPQ algorithms was done by Jochem Kuijpers et al. A set of CFPQ algorithms was implemented and evaluated using Neo4j as a graph storage. Results were presented in~\cite{Kuijpers:2019:ESC:3335783.3335791}. It was concluded that the existing algorithms are not performant enough to be used for the real-world data analysis.

Regarding graph databases, CFPQ was applied in such graph analysis related tasks as biological data analysis~\cite{SubgraphQueriesbyContextfreeGrammars}, data provenance~\cite{8731467}, hierarchy analysis in RDF data~\cite{MEDEIROS2022101089,zhangCfPgRdf}.

Multiple CFPQ algorithms are based on different parsing algorithms and techniques. For example, an approach based on parsing combinators was proposed by Ekaterina Verbitskaia et al. in~\cite{10.1145/3241653.3241655}. Several algorithms based on LL-like and LR-like techniques were developed by Ciro M. Medeiros et al. in~\cite{10.1145/3427081.3427087,MEDEIROS201975,MEDEIROS2022101089}. Also, CFPQ algorithms were investigated by Phillip Bradford in~\cite{Bradford2009,8249039} and Charles B. Ward et al. in~\cite{4625871}. An algorithm based on matrix equations was proposed by Yuliya Susanina in~\cite{10.1145/3318464.3384400}.
Paths extraction problem was studied by Jelle Hellings in~\cite{Hellings2020ExplainingPQ}.

A set of linear-algebra-based CFPQ algorithms was developed by Rustam Azimov et al., including all-path and single-path variants, proposed in~\cite{10.1145/3461837.3464513} and~\cite{10.1145/3398682.3399163}, respectively. The Kronecker product-based algorithm~\cite{OrachevKronecker} was proposed by Egor Orachev et al., and a multiple-source CFPQ algorithm for RedisGraph was proposed by Arseniy Terekhov et al. in~\cite{DBLP:conf/edbt/TerekhovPAZG21}. 

Recursive state machines were studied in the context of CFPQ in several papers, including~\cite{OrachevKronecker} where Egor Orachev et al. use RSM to specify context-free constraints, Yuxiang Lei et al.~\cite{10.1145/3591233} propose to use RSM to specify path constraints, and~\cite{10.1145/1328438.1328460} where Swarat Chaudhuri proposes a slightly subcubic algorithm for the reachability problem for recursive state machines --- the equivalent to CFPQ problem.

GLL was introduced by Elizabeth Scott and Adrian Johnstone in~\cite{SCOTT2010177}. A number of modifications of the GLL algorithm were further proposed, including the version which supports EBNF without its transformation~\cite{SCOTT2018120} and the version which uses binary subtree sets~\cite{SCOTT201963} instead of SPPF. The latest version simplifies the algorithm and avoids the overhead of the explicit graph construction. Within it, the optimized and simplified OOP-friendly version of GLL was proposed by Ali Afroozeh and Anastasia Izmaylova in~\cite{AfroozenFasterGll}.